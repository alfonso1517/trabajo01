% Options for packages loaded elsewhere
\PassOptionsToPackage{unicode}{hyperref}
\PassOptionsToPackage{hyphens}{url}
%
\documentclass[
]{article}
\usepackage{amsmath,amssymb}
\usepackage{lmodern}
\usepackage{iftex}
\ifPDFTeX
  \usepackage[T1]{fontenc}
  \usepackage[utf8]{inputenc}
  \usepackage{textcomp} % provide euro and other symbols
\else % if luatex or xetex
  \usepackage{unicode-math}
  \defaultfontfeatures{Scale=MatchLowercase}
  \defaultfontfeatures[\rmfamily]{Ligatures=TeX,Scale=1}
\fi
% Use upquote if available, for straight quotes in verbatim environments
\IfFileExists{upquote.sty}{\usepackage{upquote}}{}
\IfFileExists{microtype.sty}{% use microtype if available
  \usepackage[]{microtype}
  \UseMicrotypeSet[protrusion]{basicmath} % disable protrusion for tt fonts
}{}
\makeatletter
\@ifundefined{KOMAClassName}{% if non-KOMA class
  \IfFileExists{parskip.sty}{%
    \usepackage{parskip}
  }{% else
    \setlength{\parindent}{0pt}
    \setlength{\parskip}{6pt plus 2pt minus 1pt}}
}{% if KOMA class
  \KOMAoptions{parskip=half}}
\makeatother
\usepackage{xcolor}
\IfFileExists{xurl.sty}{\usepackage{xurl}}{} % add URL line breaks if available
\IfFileExists{bookmark.sty}{\usepackage{bookmark}}{\usepackage{hyperref}}
\hypersetup{
  pdftitle={Enunciados},
  pdfauthor={Alfonso Lopez Santiago},
  hidelinks,
  pdfcreator={LaTeX via pandoc}}
\urlstyle{same} % disable monospaced font for URLs
\usepackage[margin=1in]{geometry}
\usepackage{graphicx}
\makeatletter
\def\maxwidth{\ifdim\Gin@nat@width>\linewidth\linewidth\else\Gin@nat@width\fi}
\def\maxheight{\ifdim\Gin@nat@height>\textheight\textheight\else\Gin@nat@height\fi}
\makeatother
% Scale images if necessary, so that they will not overflow the page
% margins by default, and it is still possible to overwrite the defaults
% using explicit options in \includegraphics[width, height, ...]{}
\setkeys{Gin}{width=\maxwidth,height=\maxheight,keepaspectratio}
% Set default figure placement to htbp
\makeatletter
\def\fps@figure{htbp}
\makeatother
\setlength{\emergencystretch}{3em} % prevent overfull lines
\providecommand{\tightlist}{%
  \setlength{\itemsep}{0pt}\setlength{\parskip}{0pt}}
\setcounter{secnumdepth}{-\maxdimen} % remove section numbering
\ifLuaTeX
  \usepackage{selnolig}  % disable illegal ligatures
\fi

\title{Enunciados}
\author{Alfonso Lopez Santiago}
\date{2024-10-11}

\begin{document}
\maketitle

\#PROBLEMA 2 Problema de estrategia en un partido de fútbol

Un equipo de fútbol está en los últimos 10 minutos de un partido muy
cerrado y debe elegir una estrategia para atacar. El entrenador tiene
tres alternativas tácticas, cada una con diferentes probabilidades de
éxito dependiendo de lo que el equipo rival haga en esos momentos
decisivos.

Estrategias disponibles: 1. Atacar por las bandas 2. Atacar por el
centro 3. Jugar balones largos (pases directos a los delanteros)

Escenarios posibles: Escenario 1: El equipo rival se queda con un
jugador menos (expulsión) Escenario 2: El equipo rival juega a la
defensiva y cierra los espacios Escenario 3: El equipo rival está
buscando atacar y deja huecos en su defensa Escenario 4: El equipo rival
presiona en todo el campo (presión alta)

Probabilidades de éxito para cada estrategia en cada escenario: - Atacar
por las bandas: - Escenario 1: 85\% - Escenario 2: 40\% - Escenario 3:
80\% - Escenario 4: 50\%

\begin{itemize}
\tightlist
\item
  Atacar por el centro:

  \begin{itemize}
  \tightlist
  \item
    Escenario 1: 50\%
  \item
    Escenario 2: 30\%
  \item
    Escenario 3: 90\%
  \item
    Escenario 4: 60\%
  \end{itemize}
\item
  Jugar balones largos:

  \begin{itemize}
  \tightlist
  \item
    Escenario 1: 60\%
  \item
    Escenario 2: 70\%
  \item
    Escenario 3: 70\%
  \item
    Escenario 4: 80\%
  \end{itemize}
\end{itemize}

Pregunta: Si no se puede prever con certeza cuál será el escenario en
los últimos 10 minutos y todos los escenarios son igualmente probables,
¿qué estrategia debería usar el entrenador para maximizar las
probabilidades de éxito?

\end{document}
